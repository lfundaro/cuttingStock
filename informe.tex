\documentclass[letter,10pt]{article}

\usepackage[spanish]{babel}
\usepackage[utf8]{inputenc}
\usepackage{bookman}
\usepackage{color}
\usepackage{graphicx}
\usepackage[pdftex=true,colorlinks=true,linkcolor=black,urlcolor=blue,plainpages=false]{hyperref}


%opening
\title{Proyecto II}
\title{GRASP \& ILS}
\author{Lorenzo Fundar\'o - 0639559 \& Germán Jaber - 0639749}

\begin{document}
\begin{figure}[t]
\begin{center}
 \includegraphics[scale = 0.75]{usb.png}
\end{center}

\begin{center}
  \large Universidad Simón Bolívar
\end{center}
\begin{center}
  \large Diseño de Algoritmos II
\end{center}
\end{figure}
\maketitle

\thispagestyle{empty}
\newpage


\tableofcontents{}
\newpage

\section{Breve descripción del problema}

En teoría de complejidad computacional, el ``Bin Packing Problem'' es un problema combinatorio de dificultad NP-HARD. Se tienen objetos que
tienen asignados un costo C e infinitos contenedores idénticos con capacidad V, el problema consiste en colocar los objetos dentro de la
mínima cantidad posible.\\
\indent Existen muchas variaciones de este problema, éstas difieren, básicamente, en las dimensiones con las que se trabaja.
Entre las versiones existentes tenemos 1D, 2D, Strip Packing, Cutting Stock y 3D. Estas tienen muchas aplicaciones, como por ejemplo llenar
containers con fines de exportación, llenar camión es con una capacidad determinada de peso, crear respaldos de archivos en medios portátiles
(pendrives, disco duro portátil, etc), encontrar el orden de ejecución que menos tarde para programas que comparten memoria. Nosotros,
particularmente, abordaremos la variación 2D.\\
\indent La variaci\'on cutting stock 1-D consiste en colocar rectangulos
de altura fija y ancho variable dentro de contenedores de ancho fijo y
altura infinita. Los rect\'angulos deben estar posicionados a lo largo del
fondo del contenedor y pueden ser apilados; tambi\'en deben estar
colocados de manera ortogonal a los lados del contenedor. El objetivo es
reducir lo m\'as posible hasta que altura se usan los contenedores, o
visto de otra forma, reducir lo m\'as posible el desperdicio que se
genera a causa de los espacios que son muy peque\~nos para contener
alg\'un rect\'angulo.

\section{Descripción del problema espec\'ifico tratado}

En nuestro caso particular, tratamos el problema de cutting stock aplicado a la industria de
la fabricaci\'on de rollos de papel. En este caso particular se tienen \texttt{tipos de corte} de un rollo
que tienen asociada una demanda y un taman\~o de lote; tambi\'en se tienen varios \texttt{tipos de rollos} sobre los cuales se 
pueden hacer cortes.\\
\indent La demanda es la cantidad de cortes de ese tipo que se requieren. El tama\~no de lote es la cantidad que
\texttt{cortes} que van en un solo lote de ese tipo de corte. Cuando los rollos son cortados en una f\'abrica, las piezas resultantes
se empacan en lotes cuyo tama\~no depende de la pieza que se empaque en ese lote.\\
\indent En este caso, los cortes son los rect\'angulos que se colocan en los contenedores; los contendores son
varios rollos de un mismo tipo.\\
\indent Para resolver el problema, presentamos la soluci\'on como un conjunto de \texttt{cutting groups}, que son
conjuntos de \texttt{cortes} que se hacen sobre un mismo \texttt{tipo de rollo}. Se pueden tener varios cutting
groups que utilizen el mismo \texttt{tipo de rollo}, los \texttt{cortes} se distribuyen sobre los cutting group.\\
\indent Para nuestra formulaci\'on existen otras dos limitantes extra que consideramos y que se deben cumplir para
todos los cutting group que sean parte de la soluci\'on:
\begin{itemize}
  \item No puede haber m\'as de N \texttt{tipos de corte} en un mismo cutting group, siendo N definido por el usuario.
  \item La diferencia de longitud cualquier par de \texttt{tipo de corte} en un mismo cutting group no puede ser
    menor a M, siendo M definido por el usuario.
\end{itemize}
\indent Estas restricciones nacen de un inter\'es comercial debido a que ciertas partes de el proceso de fabricaci\'on
de rollos de papel son ejecutadas por operadores humanos.\\
\indent Un plan de corte es un conjunto de cutting groups que satisfacen de manera exacta la demanda de cada \texttt{tipo de corte}.\\
\indent Recomendamos leer el paper del cual extrajimos esta abstracci\'on particular,
particularmente las p\'aginas 2,3,4 y la figura 2 en la p\'agina 5.\\

\subsection{Especificación formal}

Dado un contenedor de capacidad V y una lista $a_1, ..., a_n$ de elementos de capacidad variable, se busca encontrar un entero
B junto con una B-partición $S_1 \cup S_B$ de $\{1,...,n\}$ tal que $\sum_{i \in S_k}^{} {a_i \le V}$ para todo $k = 1,..,B$.
Una solución es óptima si posee un B mínimo.

\section{Descripción de las heurísticas empleadas}

\subsection{Representación de la solución}
Los cutting group, los tipos de corte y los tipos de rollo se numeran todos desde el cero.\\
\indent La soluci\'on se representa como una matriz donde las columnas representan cutting groups. Las filas representan cantidad de tipos
de corte. As\'i, si la posici\'on [2,3] contiene un 7, quiere decir que en el cutting 3 hay 7 cortes del tipo 2.

\subsection{Función Objetivo}
Se trata de minizar el desperdicio total del plan de corte. Cuando en un cutting group un rollo se corta de tal manera que no se utiliza
completamente, y el resto que sobra es inutilizable porque no existe ning\'un corte:
\begin{itemize}
\item lo suficientemente pequen\~no para utilizarlo.
\item que pueda incluirse en el cutting group cumpliendo las restricciones.
\item cuya demanda no halla sido cubierta.
\end{itemize}
Decimos que ese resto es desperdicio.

\subsection{Operadores}
La vecindad de una soluci\'on son todos aquellos planes de corte donde se halla movido un tipo de pieza de un cuttin group a otro manteniendo
las restricciones. La cantidad de piezas que se mueven es igual al tama\~no de lote de esa pieza.\\
\indent Este operador se monta sobre una busqueda local primer mejor.\\
\indent Cuando se pasa una pieza de un cutting group a otro, ambos grupos son procesados por el algoritmo FFD, algoritmo greedy que busca una
forma de organizar las piezas de tal forma que usen la menor cantidad de rollos posible.\\
\indent Para decidir de que cutting group a que cutting group se mover\'a una pieza, se le asignan dos puntajes a cada cutting group, uno como
origen y otro como destino. Estos puntajes miden que tan bueno y deseable es que un cutting group dado rinda piezas (para el puntaje como origen)
y que tan bueno es como receptor de piezas (para el puntaje como destino).\\
\indent Los puntajes se calculan de la siguiente forma:

\begin{center}
Como origen

$FUNC(\alpha(1-Fragmentacion) OP \beta(1-Leftover))$
\end{center}

\begin{center}
Como destino

$FUNC(\omega(1-Fragmentacion) OP \gamma(Leftover))$
\end{center}

donde

\begin{center}
$Fragmentacion = \frac{U\_Rolls}{N\_Cuts}$
\end{center}

\begin{center}
$Leftover = \frac{CG\_LO}{Max\_LO}$
\end{center}

$\alpha$,$\beta$,$\omega$ y $\gamma$ son par\'ametros definidos por el usuario para calibrar la calidad de la funci\'on. $OP$ es un operador aritm\'etico
b\'asico que se usa para combinar las dos partes de la funci\'on, actualmente es el $+$. $FUNC$ es una funci\'on que se le aplica a todo el resultado,
actualmente no se aplica nada.\\
\indent $U\_Rolls$ denota la cantidad de rollos usados por ese cutting group y $N\_Cuts$ es el numero total de cortes en ese mismo cutting group;
$CG\_LO$ es el leftover de ese cutting group y $Max\_LO$ denota el leftover m\'as grande de entre todos los cutting group.\\
\indent La idea que un buen origen es aquel cuyas piezas estan muy fragmentadas (brinda flexibilidad) y al cual le queda poco espacio usado
(ya casi esta vac\'io). Un buen destino es aquel que esta muy fragmentado (brinda flexibilidad) y el cual esta casi lleno (se busca lo compacto).\\
\indent El algoritmo para cuando encuentra una soluci\'on que reduce el leftover total de la soluci\'on.\\
\indent La flexibilidad en este caso es importante para facilitar el trabajo de FFD.\\
\indent Luego, se prueban los mejores origenes contra todos los destinos, empezando con los peores. La idea es que un buen origen puede llenar f\'acilmente
un mal destino. Cuando un origen a sido probado contra todos los destinos, se prueban todos los destinos en el mismo orden con un origen peor.\\
\indent La idea es que se retrase lo m\'as posible el momento donde se prueban malos origenes con malos destinos.

\section{ILS}
La funci\'on de perturbaci\'on es consiste en mover de manera aleatoria piezas de un entre los cutting groups manteniendo las restricciones.
Es un cambio de vecindad aleatorio. La fuerza de la perturbaci\'on viene dada por la cantidad de cambios de vecindad que se hacen.\\
\indent Adicionalmente, cuando se genera una soluci\'on peor, se acepta con usando la f\'ormula para la probabilidad de recocido simulado.

\section{GRASP}
La construcci\'on de la soluci\'on inicial de GRASP se hace de manera muy simple. Dado un rollo actual que puede tener cortes ya asignados, 
se buscan los \texttt{k} cortes que al ser introducidos al rollo minimizen el desperdicio y se elige uno al azar. En este caso \texttt{k} es el
tama\~no de la lista RCL.\\
\indent En caso de que no se pueda introducir el corte en ese rollo, se introduce en un nuevo rollo y ese rollo pasa a ser el rollo actual. En caso
de que no se pueda introducir el corte manteniendo las restricciones sobre el cutting group, se abre un nuevo cutting group.

\section{Instancias}
Se usaron 6 instancias aleatorias generadas por una modificaci\'on de CUTGEN (Gau and Wascher, 1995).
Estas instancias, junto con muchas otras, se encuentran en http://www-sys.ist.osaka-u.ac.jp/~umetani/index-e.html.

\section{Valores iniciales}
El valor inicial utilizado fu\'e todos los cortes de un mismo tipo en un cutting group separado.

\section{Algoritmo genético}

\subsection{Representación de la solución}
La solución se codifica de la misma manera que se viene haciendo con las metaheurísticas anteriores. 
Se tiene un vector que contiene los grupos de corte, y en cada grupo están las piezas que se piden 
cortar. Un conjunto de piezas de un grupo de corte deben cumplir las restricciones del problema.

\subsection{Función de Fitness}
La función de fitness se calcula en base a dos valores:
\begin{itemize}
 \item \textbf{Desperdicio (leftover):} indica todo el espacio desperdiciado por la solución. 
  \item \textbf{Penalidad (penalty):} indica las veces que se ha debido arreglar una solución producto de un cruce.
\end{itemize}

Ambos valores son sumados para obtener el valor de fitness. En esta suma, la variable leftover aporta un 60 porciento ya que 
el desperdicio es una de las restricciones más importantes que queremos satisfacer, mientras que la variable penalidad aporta 
un 40 porciento.

\subsection{Función de Cruce}

Se utiliza el método de cruce de un punto. El cruce produce en la mayoría de las veces soluciones que no son factibles, por 
lo tanto se debe utilizar la función fixSolution para completar o quitar piezas de la solución.

\subsection{Método de selección}

El método de selección que se decidió utilizar es el de la ruleta. Para esto se tiene un vector con las probabilidades de 
cada uno de los componentes de la población actual. Esta probabilidad se calcula de la forma función de fitness de elemento 
en particular entre sumatoria de función de fitness de todos los elementos de la población.

\subsection{Estrategia de reemplazo}

Se utilizó la estrategia estacionaria. Se eliminan los individuos que presentan el peor fitness. Dicho individuo 
es reemplazado por uno de los hijos que se producen en el cruce. 

\subsection{Tamaño de la población}

El tamaño de la población varía según la constante FRAC que está en el archivo genetic.h. Sin embargo la cota es 
2 veces la cantidad de tipos de piezas del problema.

\subsection{Mutación}

La mutación se lleva a cabo con una probabilidad por debajo del 10 porciento. Consiste en tomar aleatoriamente uno de los hijos 
y de nuevo de manera aleatoria elegir una pieza, un origen y un destino para mover dicha pieza.

\subsection{Algunas recomendaciones para la corrida}

A continuación se definen las constantes que se pueden encontrar en genetic.h:

\begin{itemize}
\item NUM\_PERTURBATIONS: indica el número de perturbaciones que se le hará la solución inicial 
para así obtener un solución aleatoria.
\item MOVE\_PERCENTAGE: indica el porcentaje de piezas que se moverá de un grupo de corte a otro.
\item DIFF: es la diferencia que debe existir entre cada pieza de un grupo de corte.
\item RO: es la máxima cantidad de Open Stacks que puede haber en un momento dado.
\item CROSS\_PROB: probabilidad de cruce. El valor 0,1 indica que hay un 90 porciento de probabilidad de cruce.
\item MUTATION\_FACT: factor de mutación. 
\item FRACC: indica la fracción de población que se va a utilizar para el tamaño de la población. Por defecto 
este valor se encuentra en 1,0 por lo tanto se usa el doble de la población.
\item MAX\_IT: número máximo de iteraciones o generaciones que se harán en el algoritmo genético. Por defecto se 
 encuentra en 30 mil.
\end{itemize}

\subsection{Dificultades presentadas}

El algoritmo es funcional cuando las restriccines de Open Stacks y mínima diferencia entre piezas se relajan. 
Se intentó utilizar los valores de restricciones que aparecen en el paper adjunto con este proyecto, sin embargo,
al arreglar las soluciones no se pueden satisfacer las restricciones y el algoritmo queda en un loop infinito 
tratando de satisfacer alguna de las restricciones. Por eso se recomienda probar el algoritmo con las constantes 
RO : 100 y DIFF: 1. 

Por otra parte, la restricción de desperdicio si se logra satisfacer y de una manera bastante buena con respecto a 
los resultados de otros algoritmos. Luego se verá en las tablas los resultados obtenidos.

\subsection{¿Cómo interpretar las tablas?}

Debajo del nombre de cada instancia aparece un número con decimales. Este indica el porcentaje de pérdida que se tiene 
por cada corrida. Luego al lado viene el tiempo y finalmente el número de iteraciones utilizadas en la búsqueda de la 
solución. 

\subsection{Conclusiones}

Los resultados del Algoritmo Genético se muestran con dos restricciones del problema que han sido relajadas para porder
darle oportunidad de convergencia al algoritmo. Se presentaron problemas para satisfacer todas las restricciones cuando 
estas se hacían más exigentes. Por ejemplo, se pobró correr las instancias con RO $=$ 4 y DIFF = 100 y el algoritmo se 
quedaba ciclando infinito para tratar de satisfacer dichas restricciones. A pesar de este ajuste, el algoritmo se comporta 
muy bien para reducir la cantidad de material perdido, que es una de las restricciones más importantes del problema. 
Se probaron cuatro instancias, las cuales fueron ejecutadas 5 veces cada una. Las instancias que comienzan por ``r'' fueron 
generadas aleatoriamente, mientras que aquellas que comienzan por ``f'' son traídas de situaciones de la vida real.
La presentación de los resultados del algoritmo genético se encuentran aislados de los resultados de GRASP e ILS ya que no 
tiene mucho sentido comparar resultados entre estas metaheurísticas si las restricciones no fueron las mismas. 

\newpage
\section{Bibliograf\'ia}
\begin{itemize}
 \item K. Matsumoto, S. Umetani, H. Nagamochi. One-dimensinal cutting stock problem for a paper tube industry, 2008
 \item F. Glover, G. Kochenberger. Handbook of Metaheuristics, Kluwer Academic Publishers, 2003
 \item http://www-sys.ist.osaka-u.ac.jp/~umetani/index-e.html - Visitada el 22/06/10
 \item http://en.wikipedia.org/wiki/Cutting\_stock\_problem - Visitada el 22/06/10
 \item http://www.cplusplus.com/reference/ - Visitada el 22/06/10
 \item http://www.delorie.com/gnu/docs/gdb/gdb\_toc.html\#SEC\_Contents - Visitada el 22/06/10
\end{itemize}
\end{document}

